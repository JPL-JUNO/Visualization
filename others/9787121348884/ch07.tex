\chapter{共享绘图区域的坐标轴}
在使用 matplotlib 实践 Python 数据可视化的过程中,我们都离不开一个重要的呈现载体:画布(figure)。我们的所有数据可视化实践都是在画布上进行操作和展示的。因此,画布的有效和正确地使用就成为需要重点研究的方面。要想实现画布的合理使用,可以借助共享绘图区域的坐标轴实现。因为,坐标轴是图形的重要载体,同时也是划分画布绘图区域的有效展示工具。
\section{共享单一绘图区域的坐标轴}
通过使用子区可以在一张画布中创建多个绘图区域,然后在每个绘图区域分别绘制图形。有时候,我们又想将多张图形放在同一个绘图区域,不想在每个绘图区域只绘制一幅图形。这时候,就可以借助共享坐标轴的方法实现在一个绘图区域绘制多幅图形的目的。

\section{共享不同子区绘图区域的坐标轴}
很多时候,我们需要共享不同子区的绘图区域的坐标轴,以求强化绘图区域的展示效果,实现精简绘图区域的目的。这时,我们通过调整函数 subplots() 中的参数 sharey(或是参数 sharex)的不同取值情况,从而实现共享不同子区的绘图区域的坐标轴的需求。

具体而言,参数 sharex 和参数 sharey 的取值形式有四种,分别是 “row”,“col”,“all” 和 “none”,其中 “all” 和 “none” 分别等同于 “True” 和 “False”。
\section{共享个别子区绘图区域的坐标轴}
我们可以对个别子区做出更加细微的局部调整,以求视图展示效果更加理想和美观。
\subsection{延伸阅读——用函数 autoscale() 调整坐标轴范围}
如果我们对某一个子区的坐标轴范围和数据范围的搭配比例不是很满意,可以使用函数 autoscale() 进行坐标轴范围的自适应调整,以使图形可以非常紧凑地填充绘图区域。调用签名是
\begin{verbatim}
    autoscale(enable=True,axis="both",tight=True)
\end{verbatim}
调用签名中的具体参数的含义如下所示。
\begin{itemize}
    \item enable:进行坐标轴范围的自适应调整。
    \item axis:使 x、y 轴都进行自适应调整。
    \item tight:让坐标轴的范围调整到数据的范围上。
\end{itemize}