\chapter{使用函数绘制 matplotlib 的图表组成元素\label{ch01}}
\section{绘制 matplotlib 图表组成元素的主要函数}
在一个图形输出窗口中,底层是一个 Figure 实例,我们通常称之为画布,包含一些可见和不可见的元素。

在画布上,自然是图形,这些图形就是 Axes 实例,Axes 实例几乎包含了我们要介绍的 matplotlib组成元素,例如坐标轴、刻度、标签、线和标记等。Axes 实例有 x 轴和 y 轴属性,也就是可以使用 Axes.xaxis 和 Axes.yaxis 来控制 x 轴和 y 轴的相关组成元素,例如刻度线、刻度标签、刻度线定位器和刻度标签格式器。
\figures{fig1-1}{图的组成元素。以改图为切入点,从这些函数的函数功能、调用签名、参数说明和调用展示四个方面来全面阐述 API 函数的使用方法和技术细节。}
\section{绘制 matplotlib 图表组成元素的函数用法}
\begin{table}
    \centering
    \caption{绘制 matplotlib 图表组成元素的函数用法}
    \label{tbl1-1}
    \begin{tabular}{ll}
        \hline
        函数名        & 函数功能              \\
        \hline
        plot()     & 展现变量的趋势变化         \\
        scatter()  & 寻找变量之间的关系         \\
        xlim()     & 设置 x 轴的数值显示范围     \\
        xlabel()   & 设置 x 轴的标签文本       \\
        grid()     & 绘制刻度线的网格线         \\
        axhline()  & 绘制平行于 x 轴的水平参考线   \\
        axvspan()  & 绘制垂直于 x 轴的参考区域    \\
        annotate() & 添加图形内容细节的指向型注释文本  \\
        text()     & 添加图形内容细节的无指向型注释文本 \\
        title()    & 添加图形内容的标题         \\
        legend()   & 标示不同图形的文本标签图例     \\
        \hline
    \end{tabular}
\end{table}