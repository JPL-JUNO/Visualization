\chapter{\label{ch02}}
从基础统计图形函数的功能、调用签名、参数说明和调用展示四个层面来讲解统计函数的使用方法和参数概念,以此建立对 Python 数据可视化的直观认识。
\begin{table}[h]
    \centering
    \caption{使用统计函数绘制简单图形}
    \label{tbl2-1}
    \begin{tabularx}{\textwidth}{llX}
        \hline
        函数         & 说明       & 函数功能                       \\
        \hline
        bar()      & 用于绘制柱状图  & 在 $x$ 轴上绘制定性数据的分布特征        \\
        barh()     & 用于绘制条形图  & 在 $y$ 轴上绘制定性数据的分布          \\
        hist()     & 用于绘制直方图  & 在 $x$ 轴上绘制定量数据的分布特征        \\
        pie()      & 用于绘制饼图   & 绘制定性数据的不同类别的百分比            \\
        polar()    & 用于绘制极线图  & 在极坐标轴上绘制折线图                \\
        scatter()  & 用于绘制气泡图  & 二维数据借助气泡大小展示三维数据           \\
        stem()     & 用于绘制棉棒图  & 绘制离散有序数据                   \\
        boxplot()  & 用于绘制箱线图  & 绘制箱线图                      \\
        errorbar() & 用于绘制误差棒图 & 绘制 $y$ 轴方向或是 $x$ 轴方向的误差范围。 \\
        \hline
    \end{tabularx}
\end{table}