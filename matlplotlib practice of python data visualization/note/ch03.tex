\chapter{绘制统计图形\label{ch03}}
\section{堆积图}
\subsection{堆积柱状图}
如果将函数 bar() 中的参数 bottom 的取值设定为列表 y,函数 \verb|bar(x, y1, bottom=y, color="r")| 就会输出堆积柱状图。
\subsection{堆积条形图}
如果将函数 barh() 中的参数 left 的取值设定为列表 y,函数 \verb|barh(x, y1, left=y ,color="r")| 就会输出堆积条形图。
\section{分块图}
如果我们不将多数据以堆积图的形式进行可视化展示,那么就需要借助分块图来对比多数据的分布差异。同样,分块图可以分为多数据并列柱状图和多数据平行条形图。
\subsection{多数据并列柱状图}
对于堆积柱状图而言,我们也可以选择多数据并列柱状图来改变堆积柱状图的可视化效果。当然,堆积条形图也可以改变可视化效果,呈现多数据平行条形图的图形样式。
\section{参数探索}
如果想在柱体上绘制装饰线或装饰图,也就是说,设置柱体的填充样式。我们可以使用关键字参数 hatch,关键字参数 hatch 可以有很多取值,例如,$/$, $\backslash\backslash$, $|$, $-$等,每种符号字符串都是一种填充柱体的几何样式。而且,符号字符串的符号数量越多,柱体的几何图形的密集程度越高。(对视觉颜色障碍有好处)
\section{堆积折线图、间断条形图和阶梯图}
\subsection{用函数 stackplot() 绘制堆积折线图}
堆积折线图是通过绘制不同数据集的折线图而生成的。堆积折线图是按照垂直方向上彼此堆叠且又不相互覆盖的排列顺序,绘制若干条折线图而形成的组合图形。
\subsection{用函数 broken\_barh() 绘制间断条形图}
间断条形图是在条形图的基础上绘制而成的,\textbf{主要用来可视化定性数据的相同指标在时间维度上的指标值的变化情况,实现定性数据的相同指标的变化情况的有效直观比较}。
\subsection{用函数 step() 绘制阶梯图}
阶梯图在可视化效果上正如图形的名字那样形象,就如同山间的台阶时而上升时而下降,从图形本身而言,很像折线图。也用采是反映数据的趋势变化或是周期规律的。\textbf{阶梯图经常使用在时间序列数据的可视化任务中,凸显时序数据的波动周期和规律}。
\section{直方图}
直方图是\textbf{用来展现连续型数据分布特征的统计图形}。利用直方图我们可以直观地分析出数据的集中趋势和波动情况。
\section{饼图}
饼图是用来展示定性数据比例分布特征的统计图形。通过绘制饼图,我们可以清楚地观察出数据的占比情况。
\section{箱线图}
箱线图是由一个箱体和一对箱须所组成的统计图形。箱体是由第一四分位数、中位数(第二四分位数)和第三四分位数所组成的。在箱须的末端之外的数值可以理解成离群值,因此,箱须是对一组数据范围的大致直观描述。

\section{误差棒图}
在很多科学实验中都存在测量误差或是试验误差,这是无法控制的客观因素。这样,在可视化试验结果的时候,最好可以给试验结果增加观测结果的误差以表示客观存在的测量偏差。误差棒图就是可以运用在这一场景中的很理想的统计图形。

\subsection{应用场景——定量数据的误差范围}
通过抽样获得样本,对总体参数进行估计会由于样本的随机性导致参数估计值出现波动,因此需要用误差置信区间来表示对总体参数估计的可靠范围。误差棒就可以很好地实现充当总体参数估计的置信区间的角色。误差棒的计算方法可以有很多种:单一数值、置信区间、标准差和标准误等。误差棒的可视化展示效果也有很多种样式:水平误差棒、垂直误差棒、对称误差棒和非对称误差棒等。